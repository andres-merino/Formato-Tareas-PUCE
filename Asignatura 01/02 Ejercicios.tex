\documentclass[11pt]{article}
% -- Formato
% -- Formato realizado en febrero de 2024
% -- Corre sin problema en TeX Live 2023
% -- Adapta correctamente normas APA con BibLaTeX

% -- Paquetes básicos
\usepackage[spanish,es-nolists]{babel}
\usepackage[utf8]{inputenc}
\usepackage[noblocks]{authblk} % Para poner afiliación
    \renewcommand\Authsep{; }
    \renewcommand\Authand{; }
    \renewcommand\Authands{; }
    \renewcommand\Affilfont{\small}
    \setcounter{Maxaffil}{1}
\usepackage{xcolor}
\usepackage{booktabs}
\usepackage{float}
\usepackage{graphicx}
\usepackage[colorlinks,allcolors=.]{hyperref}
\usepackage[a4paper, hmargin=2cm, vmargin=4cm]{geometry}
\usepackage{iftex}
\usepackage{silence}

% -- Formato de título
\usepackage{titling}
\pretitle{\vspace*{-2cm}\begin{center}\bfseries\LARGE}
\posttitle{\end{center}}

% -- Tipo de letra
\ifPDFTeX % LaTeX y pdfLaTeX
    \RequirePackage[defaultfam,tabular,lining]{montserrat}
        \renewcommand*\oldstylenums[1]{{\fontfamily{Montserrat-TOsF}\selectfont #1}}
    \RequirePackage[OT1]{eulervm}
    \renewcommand{\labelitemi}{$\bullet$}
    \DeclareMathSizes{10}{10.78}{7}{7}
\else % XeLaTeX
    \RequirePackage[OT1]{eulervm}
    \RequirePackage{fontspec}
    \setmainfont{montserrat}
    \DeclareSymbolFont{operators}{\encodingdefault}{\familydefault}{m}{n}
    \DeclareMathSizes{10}{10.78}{7}{7}
    \renewcommand{\labelitemi}{$\bullet$}
\fi
\WarningFilter{latexfont}{Font shape  `T1/cmtt/regular/n'}
\WarningFilter{latexfont}{Some font shapes were not available, defaults substituted.}

% -- Fondo
\WarningsOff[everypage]
\usepackage[pages=all]{background}
\backgroundsetup{
    scale=1,color=black,opacity=1,angle=0,
    hshift=0mm,
    vshift=0mm,
    contents={\includegraphics[width=210mm]{HojaMembretada.pdf}}
}

% -- Formato de bibliografía
\usepackage{csquotes}
\usepackage[backend=biber, style=apa]{biblatex}
% -- Adaptar apa al español (2023)
    \makeatletter
    \DefineBibliographyExtras{spanish}{ 
        % Código para corregir &
        \setcounter{smartand}{1}
    	\let\lbx@finalnamedelim=\lbx@es@smartand
    	\let\lbx@finallistdelim=\lbx@es@smartand
        % Código para corregir coma final
        \renewcommand*{\apablx@ifrevnameappcomma}[2]{#2}
        \let\finalandcomma=\empty
    }
    \makeatother
    \setlength{\bibhang}{\parindent}
% -- Fin de adaptar apa al español (2023)

% -- Formato leyendas
\usepackage[font={small},labelfont={bf,small},
    justification=centerlast,tablename=Tabla]{caption}
\usepackage{listings}
    \renewcommand{\lstlistingname}{Código}

% -- Formato Código
\lstdefinestyle{mystyle}{
    backgroundcolor=\color{gray!20},   
    commentstyle=\color{teal},
    keywordstyle=\color{purple},
    stringstyle=\color{violet},
    basicstyle=\ttfamily\footnotesize,
    breaklines=true,                 
    captionpos=b,                    
    showstringspaces=false,
    linewidth=0.95\linewidth,
    literate={\ \ }{{\ }}1,
    xleftmargin=0.05\linewidth,
}
\lstset{style=mystyle}

% -- Espacio
\setlength{\parskip}{0.4\baselineskip}
\renewcommand{\baselinestretch}{1.2}


\usepackage[many]{tcolorbox}
\usepackage{amsthm}
\definecolor{colordef}{cmyk}{0.81,0.62,0.00,0.22}
\newtheoremstyle{estiloteorema}%
    {9pt}{9pt}{}{0pt}
    {\bfseries\color{\tcb@@color}}
    {}{ }
    {\textsc{\thmname{#1}\thmnumber{ #2}}\thmnote{: #3}.}
\makeatletter
%%  Keys temporales: |tipo|,|color|, |contador| e |icóno|.
\def\tcb@@tipo{}
    \tcbset{ tipo/.code = {\def\tcb@@tipo{#1} } }
\def\tcb@@contador{}
    \tcbset{ contador/.code = {\def\tcb@@contador{#1} } }
\def\tcb@@color{colordef}
    \tcbset{ color/.code = {\def\tcb@@color{#1} } }
\def\tcb@@icono{{\large\faWarning}}
    \tcbset{ icono/.code = {\def\tcb@@icono{#1} } }
\tcbset{ recuadrost/.style ={
    before skip=10pt,arc=0mm,breakable,enhanced,
    colback=\tcb@@color!7,colframe=\tcb@@color,
    boxrule=0pt,leftrule=2pt,
    top=0.5mm,bottom=0.5mm,left=2mm,right=2mm,
    fontupper=\normalsize,
    % parbox=false
    }
    }
\makeatother
\newtheorem{ejer}{Ejercicio}
    \tcolorboxenvironment{ejer}{%
        color=colordef,recuadrost,colback=colordef!7,drop fuzzy shadow
    }

% -- Paquetes adicionales
\usepackage{amsmath,amssymb,amsthm}
\usepackage{aleph-comandos}
\usepackage{enumitem}

% -- Comandos extra


% -- Datos
\title{Deber de la semana 1}
\author{Nombres y apellidos}
\affil{Carrera de Ciencia de Datos}
\date{\today}


%%%%%%%%%%%%%%%%%%%%%%%%%%%%%%%%%%%%%%%%
\begin{document}

%%%%%%%%%%%%%%%%%%%%%%%%%%%%%%%%%%%%%%%%
\maketitle

%%%%%%%%%%%%%%%%%%%%%%%%%%%%%%%%%%%%%%%%
\section{Ejercicios de matrices}
%%%%%%%%%%%%%%%%%%%%%%%%%%%%%%%%%%%%%%%%


%%%%%%%%%%%%%%%%%%%%%%%%%%%%%%%%%%%%%%%%
%% Matriz inversa
%%%%%%%%%%%%%%%%%%%%%%%%%%%%%%%%%%%%%%%%
\begin{ejer}
    Considere la siguiente matriz de $3\times 3$:
    \[
        A = \begin{pmatrix}
            1 & 2 & 3 \\
            3 &-2 & 1 \\
            4 & 1 & 1
        \end{pmatrix}.
    \]
    \begin{enumerate}[leftmargin=*,label=\textit{\alph*})]
        \item
            Calcule el determinante de $A$ (puede utilizar directamente Python). ¿Por qué la matriz tiene inversa? 
        \item
            Utilizando el método de Gauss-Jordan, calcule la matriz inversa de $A$, indicando cada paso (puede utilizar directamente Python para las operaciones por filas).
    \end{enumerate}
\end{ejer}

\begin{proof}\hspace{0pt}
    \begin{enumerate}[leftmargin=*,label=\textit{\alph*})]
    \item El determinante de $A$ está dado por 
    \begin{align*}
        det(A) & =\begin{vmatrix}
            1 & 2 & 3 \\
            3 &-2 & 1 \\
            4 & 1 & 1
        \end{vmatrix} \\
        & = 1 \begin{vmatrix}
                -2 & 1 \\
                1 & 1
                \end{vmatrix}-2\begin{vmatrix}
                3 & 1 \\
                4 & 1
                \end{vmatrix} + 3 \begin{vmatrix}
                    3 & -2 \\
                    4 & 1
                \end{vmatrix}\\
        & = -3 + 2 + 33 = 32
    \end{align*}
    Por lo tanto, $A$ es invertible pues $\det(A)\neq 0$.
    \item
    Puesto que $A$ es invertible, se tiene que 
    \[
        (A|I_3)\sim (I_3|A^{-1}).
    \]
    Por lo tanto, aplicando operaciones por filas
    \begin{align*}
        \begin{pmatrix}
            1 & 2 & 3 & | & 1 & 0 & 0\\[0.75em]
            3 & -2 & 1 & | & 0 & 1 & 0\\[0.75em]
            4 & 1 & 1 & | & 0 & 0 & 1
        \end{pmatrix}& \sim \begin{pmatrix}
            1 & 2 & 3 & | & 1 & 0 & 0\\[0.75em]
            0 & -8 & -8 & | & -3 & 1 & 0\\[0.75em]
            4 & 1 & 1 & | & 0 & 0 & 1
        \end{pmatrix} && F_2 - 3F_1 \rightarrow F_2,  \\[0.25em]
            &\sim \begin{pmatrix}
            1 & 2 & 3 & | & 1 & 0 & 0\\[0.75em]
            0 & -8 & -8 & | & -3 & 1 & 0\\[0.75em]
            0 & -7 & -11 & | & -4 & 0 & 1
        \end{pmatrix} && F_3 - 4F_1 \rightarrow F_3, \\[0.25em]
            & \sim \begin{pmatrix}
            1 & 2 & 3 & | & 1 & 0 & 0\\[0.75em]
            0 & 1 & 1 & | & \dfrac{3}{8} & -\dfrac{1}{8} & 0\\[0.75em]
            0 & -7 & -11 & | & -4 & 0 & 1
        \end{pmatrix} && -\dfrac{1}{8}F_2 \rightarrow F_2, \\[0.25em]
            & \sim \begin{pmatrix}
            1 & 2 & 3 & | & 1 & 0 & 0\\[0.75em]
            0 & 1 & 1 & | & \dfrac{3}{8} & -\dfrac{1}{8} & 0\\[0.75em]
            0 & 7 & 11 & | & 4 & 0 & -1
        \end{pmatrix} && -F_3 \rightarrow F_3 \\[0.75em]
            & \sim \begin{pmatrix}
            1 & 0 & 1 & | & \dfrac{1}{4} & \dfrac{1}{4} & 0\\[0.75em]
            0 & 1 & 1 & | & \dfrac{3}{8} & -\dfrac{1}{8} & 0\\[0.75em]
            0 & 7 & 11 & | & 4 & 0 & -1
        \end{pmatrix} && F_1 - 2 F_2 \rightarrow F_1 \\[0.75em]
            & \sim \begin{pmatrix}
            1 & 0 & 1 & | & \dfrac{1}{4} & \dfrac{1}{4} & 0\\[0.75em]
            0 & 1 & 1 & | & \dfrac{3}{8} & -\dfrac{1}{8} & 0\\[0.75em]
            0 & 0 & 4 & | & \dfrac{11}{8} & \dfrac{7}{8} & -1
        \end{pmatrix} && F_3 - 7 F_2 \rightarrow F_3 \\[0.75em]
            & \sim \begin{pmatrix}
            1 & 0 & 1 & | & \dfrac{1}{4} & \dfrac{1}{4} & 0\\[0.75em]
            0 & 1 & 1 & | & \dfrac{3}{8} & -\dfrac{1}{8} & 0\\[0.75em]
            0 & 0 & 1 & | & \dfrac{11}{32} & \dfrac{7}{32} & -\dfrac{1}{4}
        \end{pmatrix} && \dfrac{1}{4}F_3 \rightarrow F_3 \\[0.75em]
            & \sim \begin{pmatrix}
            1 & 0 & 1 & | & \dfrac{1}{4} & \dfrac{1}{4} & 0\\[0.75em]
            0 & 1 & 0 & | & \dfrac{1}{32} & -\dfrac{11}{32} & \dfrac{1}{4}\\[0.75em]
            0 & 0 & 1 & | & \dfrac{11}{32} & \dfrac{7}{32} & -\dfrac{1}{4}
        \end{pmatrix} && F_2-F_3 \rightarrow F_2 \\[0.75em]
            & \sim \begin{pmatrix}
            1 & 0 & 0 & | & -\dfrac{3}{32} & \dfrac{1}{32} & \dfrac{1}{4}\\[0.75em]
            0 & 1 & 0 & | & \dfrac{1}{32} & -\dfrac{11}{32} & \dfrac{1}{4}\\[0.75em]
            0 & 0 & 1 & | & \dfrac{11}{32} & \dfrac{7}{32} & -\dfrac{1}{4}
        \end{pmatrix} && F_1-F_3 \rightarrow F_1.
        \end{align*}
    Por lo tanto, la matriz inversa de la matriz $A$ está dada por
    \[
        A^{-1} = \begin{pmatrix}
             -\dfrac{3}{32} & \dfrac{1}{32} & \dfrac{1}{4}\\[0.75em]
             \dfrac{1}{32} & -\dfrac{11}{32} & \dfrac{1}{4}\\[0.75em]
             \dfrac{11}{32} & \dfrac{7}{32} & -\dfrac{1}{4}
        \end{pmatrix}. \qedhere
    \]
\end{enumerate}
\end{proof}    


\end{document}

